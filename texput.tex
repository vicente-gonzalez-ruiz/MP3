\title{MPEG Audio~\cite{sayood2017introduction}}
\author{Vicente González Ruiz}
\maketitle
\tableofcontents

\section{Intro}
\begin{itemize}
\item Input: a sequence of 16-bit PCM samples.

\item Output: a sequence of MPEG Audio frames (frame = header +
  code-stream) which can be streamed.

\begin{verbatim}
audio   +-----------+       +--------------+         +----------+
channel | Time      |       | Quantization |         |          | MPEG audio
---+--->| frequency |---+-->| and          |-------> | Framming |------->
   |    | mapping   |   |   | coding       |         |          |
   |    +-----------+   |   +--------------+         +----------+
   |                    |           ^
   |                    |           |
   |                    |   +--------------+
   |                    +-->| Phycho-      |
   |                        | acustic      |
   +----------------------->| model        |
                            +--------------+
\end{verbatim}

\item The MPEG audio bitstream definition is normative. Most guidance
  about encoding is informative. Thus, two MPEG-compliant bitstreams
  that encode the same audio material at the same rate but on
  different encoders may sound very different. On the other hand, a
  given MPEG bitstream decoded on different decoders will result in
  essentially the same output.
\end{itemize}

\section{\href{https://en.wikipedia.org/wiki/MPEG-1_Audio_Layer_I}{Layer I}}
\begin{itemize}
\item 4:1 compression (384 kbps).
\item CBR (Constant Bit-Rate).
\end{itemize}

\subsection{Encoder}
\begin{enumerate}
\item Split $s[n]$ into blocks of $12\times 32=384$ samples. For each
  block:
  \begin{enumerate}
   \item Analyze the block using a 32-band equally-spaced (analysis)
     filter bank, producing $12$ coeffs/subband (the coeffs are
     downsampled (subsampled, decimated) by factor of $32$). $342$
     time-domain samples are transformed into $32$ subbands with $12$
     coeffs. Notice that in a subband, each coeff can be considered as
     a sample of such subband.
   \item Scale each block of $12$ coeffs to ensure that the entire
     range of the selected quantizer will be used. Output the
     *scalefactor*.
   \item Using the
     \href{https://en.wikipedia.org/wiki/Fast_Fourier_transform}{FFT},
     compute the ATH for the block (considering the masking effects).
   \item Let $R^*$ the bit-rate selected by the user. While the
     generated bit-rate $R\leq R^*$:
     \begin{enumerate}
     \item Decrement the quantization step $\Delta_b$ for each subband
       $b$, proportionally to the ATH in $b$. Compute $R$. The
       bit-rate is controlled be switching between quantizers with
       different number of bits.
     \end{enumerate}
   \item Output $\{\Delta_b\}_{b=1}^{32}$ and the quantization
     indexes.
  \end{enumerate}
\end{enumerate}

\subsection{Decoder}
\begin{enumerate}
\item For each input frame:
  \begin{enumerate}
   \item "Dequantize" the coeffs of each subband.
   \item Descale the coeffs to their original dynamic range.
   \item Apply the 32-band synthesis filters bank.
  \end{enumerate}
\end{enumerate}

\section{Loss information analysis}
\begin{itemize}
  \tightlist
\item Aliasing in the 32-band analysis filter bank.
\begin{verbatim}
   ^ Amplitude
   |     _______     _______     ____....__     _______
   |    /       \   /       \   /          \   /       \
   |   /         \ /         \ /            \ /         \
   |  / subband 0 X subband 1 X              X  sub. 31  \
   +-/-----------/-\---------/-\-----....---/-\-----------\-> frequency
\end{verbatim}
\item Quantization.
\end{itemize}

\section{\href{https://en.wikipedia.org/wiki/MPEG-1_Audio_Layer_II)}{Layer II}}
\begin{itemize}
\item Backward compatibile with MP1.
\item 8:1 compression (174 kbps).
\item CBR (Constant Bit-Rate).
\item Increases block-size to $3\times 12\times 32=1152$ samples.
\end{itemize}

\section{\href{https://en.wikipedia.org/wiki/MP3}{Layer III}}
\begin{itemize}
\item \href{https://www.xataka.com/historia-tecnologica/la-historia-del-mp3-el-formato-que-tras-casi-morir-dos-veces-revoluciono-el-mundo-de-la-musica}{``Rescued'' by Napster}.
\item Backward compatibile with MP1 and MP2.
\item CBR and VBR (Variable Bit-Rate). In this last case, users
  usually select the average bit-rate.
\item Typically, virtually lossless at 128 kbps for most human beings.
\item Improved subband analysis by means of the MDCT (using 32
  subbands, the low-frequency ones contains more than un bark, which
  generates a poor frequency resolution in the ATH computation).
\end{itemize}

\subsection{Encoder}
\begin{enumerate}
\item Split $s[n]$ into blocks of $36\times 32=1152$ samples. For each
  block:
  \begin{enumerate}
  \item Performs FFT of the block to compute the ATH and windows
    sequence.
  \item Analyze the block using a 32-band equally-spaced (analysis)
    filter bank, producing $36$ coeffs/subband.
  \item For each subband:
    \begin{enumerate}
    \item Analyze transients. If detected, use a sequece of
      start/short*3/stop windows. Otherwise, use a long window.
      \svg{windows}{800}
    \item Compute
      \href{https://en.wikipedia.org/wiki/Modified_discrete_cosine_transform}{MDCT}. This
      produces $36$ (long), $30$ (start/stop) or $12$ coeffs/subband
      (short). This step produces $18$ coeffs/subband (long), $15$
      coeffs/subband (start/stop) and $6$ coeffs/subband (short).
      \item Apply scalefactors to optimize quantization.
     \end{enumerate}
   \item Distortion control loop: keep (as much as possible) the
     quantization error below the ATH.
     \begin{enumerate}
     \item Rate control loop: Let $R^*$ the bit-rate selected by the
       user. While the generated bit-rate $R\leq R^*$:
       \begin{enumerate}
       \item Decrement the quantization step $\Delta_b$ for each
         subband $b$, proportionally to the ATH in $b$. Compute $R$
         after encoding the quantizer indexes with (static) Huffman
         coding. As in previous layers, a quantizer is selected from a
         list of predefined logaritmic quantizers.
       \end{enumerate}
     \end{enumerate}
  \end{enumerate}
\end{enumerate}

\subsection{}
\begin{enumerate}
\item For each input frame:
  \begin{enumerate}
  \item Decode the Huffman codes.
   \item ``Dequantize'' the coeffs of each subband.
   \item Descale the coeffs to their original dynamic range.
   \item Apply inverse MDCT.
   \item Apply the 32-band synthesis filters bank.
  \end{enumerate}
\end{enumerate}

\bibliography{data-compression}
